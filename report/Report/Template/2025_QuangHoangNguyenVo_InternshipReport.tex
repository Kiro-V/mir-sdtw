
%%%%%%%%%%%%%%%%%%%%%%%%%%%%%%%%%%%%%%%%%%%%%%%%%%%%%%%%%%%%%%%%%%%%%%%%%%%%%%%%%%%%%%%%%%%%%
% Setup title, etc in config.tex
%%%%%%%%%%%%%%%%%%%%%%%%%%%%%%%%%%%%%%%%%%%%%%%%%%%%%%%%%%%%%%%%%%%%%%%%%%%%%%%%%%%%%%%%%%%%%

%%%%%%%%%%%%%%%%%%%%%%%%%%%%%%%%%%%%%%%%%%%%%%%%%%%%%%%%%%%%%%%%%%%%%%%%%%%%%%%%%%%%%%%%%%%%%
%Page Layout
%%%%%%%%%%%%%%%%%%%%%%%%%%%%%%%%%%%%%%%%%%%%%%%%%%%%%%%%%%%%%%%%%%%%%%%%%%%%%%%%%%%%%%%%%%%%%


\def\myPageLayout{twoside}
%\def\myPageLayout{oneside}


%%%%%%%%%%%%%%%%%%%%%%%%%%%%%%%%%%%%%%%%%%%%%%%%%%%%%%%%%%%%%%%%%%%%%%%%%%%%%%%%%%%%%%%%%%%%%
% Personal data and user ad-hoc commands
%%%%%%%%%%%%%%%%%%%%%%%%%%%%%%%%%%%%%%%%%%%%%%%%%%%%%%%%%%%%%%%%%%%%%%%%%%%%%%%%%%%%%%%%%%%%%
\newcommand{\myTitle}{{\LARGE \sffamily{} \textbf{Report Title\\
Training with Unaligned Dataset:\\
\vspace{0.25cm}
Soft Dynamic Time Warping}}}


\newcommand{\myName}{Quang Hoang Nguyen Vo}
\newcommand{\myProf}{Prof.\ Dr.\ Meinard M\"uller}
\newcommand{\mySupervisor}{Msc.\ Johannes Zeitler}
\newcommand{\myTime}{\selectlanguage{USenglish}\today}

%%%%%%%%%%%%%%%%%%%%%%%%%%%%%%%%%%%%%%%%%%%%%%%%%%%%%%%%%%%%%%%%%%%%%%%%%%%%%%%%%%%%%%%%%%%%%
% Linespacing
%%%%%%%%%%%%%%%%%%%%%%%%%%%%%%%%%%%%%%%%%%%%%%%%%%%%%%%%%%%%%%%%%%%%%%%%%%%%%%%%%%%%%%%%%%%%%
%\def\mySpacing{\singlespacing}
%\def\mySpacing{\doublespacing}
\def\mySpacing{\onehalfspacing}
                     
\documentclass[a4paper,11pt,\myPageLayout]{book}
\usepackage[utf8]{inputenc}
\usepackage{captionSmall}

\input{config_packages_macros}

%%%%%%%%%%%%%%%%%%%%%%%%%%%%%%%%%%%%%%%%%%%%%%%%%%%%%%%%%%%%%%%%%%%%%%%%%%%%%%%%%%%%%%%%%%%%%
% Start of document
%%%%%%%%%%%%%%%%%%%%%%%%%%%%%%%%%%%%%%%%%%%%%%%%%%%%%%%%%%%%%%%%%%%%%%%%%%%%%%%%%%%%%%%%%%%%%
\begin{document}
\fancypagestyle{plain}{\pagestyle{mine}} % remove this if you don't want 
                                         % headings on the first page of a chapter
\frontmatter
\newpage

%%%%%%%%%%%%%%%%%%%%%%%%%%%%%%%%%%%%%%%%%%%%%%%%%%%%%%%%%%%%%%%%%%%%%%%%%%%%%%%%%%%%%%%%%%%%%
% Title Page
%%%%%%%%%%%%%%%%%%%%%%%%%%%%%%%%%%%%%%%%%%%%%%%%%%%%%%%%%%%%%%%%%%%%%%%%%%%%%%%%%%%%%%%%%%%%%
\input{titlepage}
\cleardoublepage{}

\pagenumbering{roman}
\pagestyle{mine}
\newpage

\section*{Abstract}
The evolution of Deep Neural Networks (DNNs) has shifted the paradigm of music information retrieval (MIR) from heuristic and mathematical models to data-driven approaches, which rely on large amounts of labelled training data.
However, it introduces challenges when training with weakly aligned datasets. In this project, we investigate the characteristics of differential dynamic time warping (dDTW) through the soft-DTW (sDTW) algorithm when training with weakly aligned data.
The main objective is to integrate soft-DTW as a loss function in the training process of a template-based chord recognition model.
The dataset will have its chord label timestamps distorted or removed to simulate weakly or unaligned data.
The dDTW loss function will then be used to train the model with the distorted dataset. The results will be compared with those obtained using the original dataset and the Connectionist Temporal Classification (CTC) loss function.
Additional tasks may include experimenting and evaluating the performance of sDTW with different stablizing strategies.
\mainmatter{}
%%%%%%%%%%%%%%%%%%%%%%%%%%%%%%%%%%%%%%%%%%%%%%%%%%%%%%%%%%%%%%%%%%%%%%%%%%%%%%%%%%%%%%%%%%%%%
% Table of Contents
%%%%%%%%%%%%%%%%%%%%%%%%%%%%%%%%%%%%%%%%%%%%%%%%%%%%%%%%%%%%%%%%%%%%%%%%%%%%%%%%%%%%%%%%%%%%%

\setcounter{tocdepth}{1}
\setcounter{page}{1}
{\parskip=0mm \tableofcontents}
% \thispagestyle{empty}

\pagestyle{mine}

%%%%%%%%%%%%%%%%%%%%%%%%%%%%%%%%%%%%%%%%%%%%%%%%%%%%%%%%%%%%%%%%%%%%%%%%%%%%%%%%%%%%%%%%%%%%%
% Chapters
%%%%%%%%%%%%%%%%%%%%%%%%%%%%%%%%%%%%%%%%%%%%%%%%%%%%%%%%%%%%%%%%%%%%%%%%%%%%%%%%%%%%%%%%%%%%%
\cleardoublepage{}

%%%%%%%%%%%%%%%%%%%%%%%%%%%%%%%%%%%%%%%%%%%%%%%%%%%%%%%%%%%%%%%%%%%%%%%%%%%%%%%%%%%%%%%
\chapter{Introduction}
\label{chapter:Introduction}
%%%%%%%%%%%%%%%%%%%%%%%%%%%%%%%%%%%%%%%%%%%%%%%%%%%%%%%%%%%%%%%%%%%%%%%%%%%%%%%%%%%%%%%

Generally speaking, the introduction chapter should introduce the 
topic of the thesis and motivate the importance of it. Moreover, the
introduction should give an outline of the thesis and point out the 
contributions of this work.

You can logically group the chapters of the thesis
by using so-called parts. An example of how to insert a part-page 
containing a teaser-image is included in this template.
Typically, you will have an introductory chapter that gives a
broad overview. Then, the first part of the thesis might start after 
the introduction.


%%%%%%%%%%%%%%%%%%%%%%%%%%%%%%%%%%%%%%%%%%%%%%%%%%%%%%%%%%%%%%%%%%%%%%%%%%%%%%%%%%%%%%%
\chapter{Soft Dynamic Time Warping Algorithm}
\label{chapter:sdtw}
%%%%%%%%%%%%%%%%%%%%%%%%%%%%%%%%%%%%%%%%%%%%%%%%%%%%%%%%%%%%%%%%%%%%%%%%%%%%%%%%%%%%%%%



%%%%%%%%%%%%%%%%%%%%%%%%%%%%%%%%%%%%%%%%%%%%%%%%%%%%%%%%%%%%%%%%%%%%%%%%%%%%%%%%%%%%%%%
\section{Forward Pass}
\label{section:forward_pass}
%%%%%%%%%%%%%%%%%%%%%%%%%%%%%%%%%%%%%%%%%%%%%%%%%%%%%%%%%%%%%%%%%%%%%%%%%%%%%%%%%%%%%%%



%%%%%%%%%%%%%%%%%%%%%%%%%%%%%%%%%%%%%%%%%%%%%%%%%%%%%%%%%%%%%%%%%%%%%%%%%%%%%%%%%%%%%%%
\section{Backward Pass}
\label{section:backward_pass}
%%%%%%%%%%%%%%%%%%%%%%%%%%%%%%%%%%%%%%%%%%%%%%%%%%%%%%%%%%%%%%%%%%%%%%%%%%%%%%%%%%%%%%%

%%%%%%%%%%%%%%%%%%%%%%%%%%%%%%%%%%%%%%%%%%%%%%%%%%%%%%%%%%%%%%%%%%%%%%%%%%%%%%%%%%%%%%%
\section{Tables}
\label{section:tables}
%%%%%%%%%%%%%%%%%%%%%%%%%%%%%%%%%%%%%%%%%%%%%%%%%%%%%%%%%%%%%%%%%%%%%%%%%%%%%%%%%%%%%%%

\begin{table}[t]
\centering
\renewcommand{\arraystretch}{1.3}
\begin{tabular}{@{}cll@{}} 
\toprule
Non-English or Math & Frequency & Comments\\ 
\midrule
\O & 1 in 1,000& For Swedish names\\
$\pi$ & 1 in 5& Common in math\\
\$ & 4 in 5 & Used in business\\
$\Psi^2_1$ & 1 in 40,000& Unexplained usage\\
\bottomrule
\end{tabular}
\vspace*{-0cm}
\caption{Frequency of Special Characters. Note that this table does not contain any
vertical lines which makes the table look more tidy.}
\label{table:characters}
\end{table}


%%%%%%%%%%%%%%%%%%%%%%%%%%%%%%%%%%%%%%%%%%%%%%%%%%%%%%%%%%%%%%%%%%%%%%%%%%%%%%%%%%%%%%%
\chapter{Experimental Setup}
\label{chapter:experimental_setup}
%%%%%%%%%%%%%%%%%%%%%%%%%%%%%%%%%%%%%%%%%%%%%%%%%%%%%%%%%%%%%%%%%%%%%%%%%%%%%%%%%%%%%%%

\section{Data Preparation}
\label{section:data_preparation}
For the experiments, we will use "the Beatles" dataset taken from Isophonics \cite{Isophonics}, which contains audio recordings and their corresponding chord annotations. This dataset is denoted as "strongly-aligned" since the chord labels are precisely aligned with the audio.
Based on this dataset, we create a "soft" version by removing the repetitions in the chord annotations, resulting in a shorter chord label sequence, as illustrated in Figure \ref{fig:strongly_vs_weakly_aligned}. This simulates a weakly-aligned dataset where the exact timing of chord changes is unknown.


\section{Model Architecture}
\label{section:model_architecture}
Given the aim of this experiment is to evaluate the performance of the proposed SDTW loss function, the network architecture plays a minor role and are kept simple. Therefore, we used a basic chord recognition model (dChord) consisting of
a single convolutional layer followed by a softmax layer. Preceding this layer is a log-compression layer to reduce the dynamic range of the input features and a feature normalization layer to standardize the input features. The model takes as input a sequence of chroma features extracted from the audio signal and outputs a hot encoded vector representing the predicted chord.

\section{Training Procedure}
\label{section:training_procedure}

The training procedure will involve optimizing the model parameters using the dDTW loss function on the distorted dataset. The model will be evaluated on both the distorted and original datasets to assess its robustness to label noise. Hyperparameter tuning and regularization techniques may be applied to improve generalization.

\section{Results}
\label{section:results}

\section{Discussion}
\label{section:discussion}

In this section, we will discuss the implications of the results obtained from the experiments. We will analyze the performance of the proposed method in comparison to existing approaches and highlight its strengths and weaknesses. Additionally, we will explore potential avenues for future research and improvements.

%%%%%%%%%%%%%%%%%%%%%%%%%%%%%%%%%%%%%%%%%%%%%%%%%%%%%%%%%%%%%%%%%%%%%%%%%%%%%%%%%%%%%%%
\chapter{Conclusions}
\label{chapter:conclusions}
%%%%%%%%%%%%%%%%%%%%%%%%%%%%%%%%%%%%%%%%%%%%%%%%%%%%%%%%%%%%%%%%%%%%%%%%%%%%%%%%%%%%%%%
\section{Limitation}
\label{section:limitations}
\section{Future Work}
\label{section:future_work}
Draw the conclusions in the big picture of the thesis! Then, indicate future work.

% %%%%%%%%%%%%%%%%%%%%%%%%%%%%%%%%%%%%%%%%%%%%%%%%%%%%%%%%%%%%%%%%%%%%%%%%%%%%%%%%%%%%%%%%%%%%%
% % Appendix
% %%%%%%%%%%%%%%%%%%%%%%%%%%%%%%%%%%%%%%%%%%%%%%%%%%%%%%%%%%%%%%%%%%%%%%%%%%%%%%%%%%%%%%%%%%%%%
% \cleardoublepage{}
% \appendix
% %%%%%%%%%%%%%%%%%%%%%%%%%%%%%%%%%%%%%%%%%%%%%%%%%%%%%%%%%%%%%%%%%%%%%%%%%%%%%%%%%%%%%%%
% \chapter{Source Code}
% \label{chapter:source_code}
% %%%%%%%%%%%%%%%%%%%%%%%%%%%%%%%%%%%%%%%%%%%%%%%%%%%%%%%%%%%%%%%%%%%%%%%%%%%%%%%%%%%%%%%

% In this chapter, the headers of selected \MATLAB{} functions created during the writing of this thesis are reproduced. The headers contain information about the name of the described function and its input/output behavior.

% \section*{Feature Extraction}
% The \texttt{file\_to\_feature} function is used as a wrapper for several low-level functions that perform feature extraction or loading of precomputed features.

% Sample usage:\\
% \scriptsize
% \verb|[f_pitch, f_peaks] = file_to_feature('features', 'pathetique.wav');|

% \begin{verbatim}
% %%%%%%%%%%%%%%%%%%%%%%%%%%%%%%%%%%%%%%%%%%%%%%%%%%%%%%%%%%%%%%%%%%%%%%%%%%%
% % Name: file_to_feature
% % Version: 1.0
% % Date: 11.05.2010
% % Programmer: John Q. Public
% %
% % Description:
% %   Load or compute features for audio and MIDI files
% %
% % Input:
% % - dirname: Directory where the file or features are located
% % - filename: Name of the file for which to load/compute features
% % - parameter
% %            .win_len: Window length used for STMSP feature generation
% %            .win_res: Window resolution
% %
% % Output:
% % - f_pitch: Pitch features (STMSP)
% % - f_peaks: Energy peaks for onset computation
% % - f_onsets: Precise onsets (only generated in case of MIDI input data)
% %%%%%%%%%%%%%%%%%%%%%%%%%%%%%%%%%%%%%%%%%%%%%%%%%%%%%%%%%%%%%%%%%%%%%%%%%%%
% \end{verbatim}
% \normalsize

%%%%%%%%%%%%%%%%%%%%%%%%%%%%%%%%%%%%%%%%%%%%%%%%%%%%%%%%%%%%%%%%%%%%%%%%%%%%%%%%%%%%%%%%%%%%
% Bibliography
%%%%%%%%%%%%%%%%%%%%%%%%%%%%%%%%%%%%%%%%%%%%%%%%%%%%%%%%%%%%%%%%%%%%%%%%%%%%%%%%%%%%%%%%%%%%
\cleardoublepage{}
{
\small
\addcontentsline{toc}{chapter}{Bibliography}
\bibliographystyle{siam}
\bibliography{bibliography}
}
\cleardoublepage{}


\end{document}
