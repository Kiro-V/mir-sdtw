
%%%%%%%%%%%%%%%%%%%%%%%%%%%%%%%%%%%%%%%%%%%%%%%%%%%%%%%%%%%%%%%%%%%%%%%%%%%%%%%%%%%%%%%%%%%%%
% Setup title, etc in config.tex
%%%%%%%%%%%%%%%%%%%%%%%%%%%%%%%%%%%%%%%%%%%%%%%%%%%%%%%%%%%%%%%%%%%%%%%%%%%%%%%%%%%%%%%%%%%%%

%%%%%%%%%%%%%%%%%%%%%%%%%%%%%%%%%%%%%%%%%%%%%%%%%%%%%%%%%%%%%%%%%%%%%%%%%%%%%%%%%%%%%%%%%%%%%
%Page Layout
%%%%%%%%%%%%%%%%%%%%%%%%%%%%%%%%%%%%%%%%%%%%%%%%%%%%%%%%%%%%%%%%%%%%%%%%%%%%%%%%%%%%%%%%%%%%%


\def\myPageLayout{twoside}
%\def\myPageLayout{oneside}


%%%%%%%%%%%%%%%%%%%%%%%%%%%%%%%%%%%%%%%%%%%%%%%%%%%%%%%%%%%%%%%%%%%%%%%%%%%%%%%%%%%%%%%%%%%%%
% Personal data and user ad-hoc commands
%%%%%%%%%%%%%%%%%%%%%%%%%%%%%%%%%%%%%%%%%%%%%%%%%%%%%%%%%%%%%%%%%%%%%%%%%%%%%%%%%%%%%%%%%%%%%
\newcommand{\myTitle}{{\LARGE \sffamily{} \textbf{Report Title\\
Training with Unaligned Dataset:\\
\vspace{0.25cm}
Soft Dynamic Time Warping}}}


\newcommand{\myName}{Quang Hoang Nguyen Vo}
\newcommand{\myProf}{Prof.\ Dr.\ Meinard M\"uller}
\newcommand{\mySupervisor}{Msc.\ Johannes Zeitler}
\newcommand{\myTime}{\selectlanguage{USenglish}\today}

%%%%%%%%%%%%%%%%%%%%%%%%%%%%%%%%%%%%%%%%%%%%%%%%%%%%%%%%%%%%%%%%%%%%%%%%%%%%%%%%%%%%%%%%%%%%%
% Linespacing
%%%%%%%%%%%%%%%%%%%%%%%%%%%%%%%%%%%%%%%%%%%%%%%%%%%%%%%%%%%%%%%%%%%%%%%%%%%%%%%%%%%%%%%%%%%%%
%\def\mySpacing{\singlespacing}
%\def\mySpacing{\doublespacing}
\def\mySpacing{\onehalfspacing}
                     
\documentclass[a4paper,11pt,\myPageLayout]{book}
\usepackage[utf8]{inputenc}
\usepackage{captionSmall}

\input{config_packages_macros}

%%%%%%%%%%%%%%%%%%%%%%%%%%%%%%%%%%%%%%%%%%%%%%%%%%%%%%%%%%%%%%%%%%%%%%%%%%%%%%%%%%%%%%%%%%%%%
% Start of document
%%%%%%%%%%%%%%%%%%%%%%%%%%%%%%%%%%%%%%%%%%%%%%%%%%%%%%%%%%%%%%%%%%%%%%%%%%%%%%%%%%%%%%%%%%%%%
\begin{document}
\fancypagestyle{plain}{\pagestyle{mine}} % remove this if you don't want 
                                         % headings on the first page of a chapter
\frontmatter
\newpage

%%%%%%%%%%%%%%%%%%%%%%%%%%%%%%%%%%%%%%%%%%%%%%%%%%%%%%%%%%%%%%%%%%%%%%%%%%%%%%%%%%%%%%%%%%%%%
% Title Page
%%%%%%%%%%%%%%%%%%%%%%%%%%%%%%%%%%%%%%%%%%%%%%%%%%%%%%%%%%%%%%%%%%%%%%%%%%%%%%%%%%%%%%%%%%%%%
\input{titlepage}
\cleardoublepage{}

\pagenumbering{roman}
\pagestyle{mine}
\newpage

\section*{Abstract}
The evolution of Deep Neural Networks (DNNs) has shifted the paradigm of music information retrieval (MIR) from heuristic and mathematical models to data-driven approaches, which rely on large amounts of labelled training data.
However, it introduces challenges when training with weakly aligned datasets. In this project, we investigate the characteristics of differential dynamic time warping (dDTW) through the soft-DTW (sDTW) algorithm when training with weakly aligned data.
The main objective is to integrate soft-DTW as a loss function in the training process of a template-based chord recognition model.
The dataset will have its chord label timestamps distorted or removed to simulate weakly or unaligned data.
The dDTW loss function will then be used to train the model with the distorted dataset. The results will be compared with those obtained using the original dataset and the Connectionist Temporal Classification (CTC) loss function.
Additional tasks may include experimenting and evaluating the performance of sDTW with different stablizing strategies.
\mainmatter{}
%%%%%%%%%%%%%%%%%%%%%%%%%%%%%%%%%%%%%%%%%%%%%%%%%%%%%%%%%%%%%%%%%%%%%%%%%%%%%%%%%%%%%%%%%%%%%
% Table of Contents
%%%%%%%%%%%%%%%%%%%%%%%%%%%%%%%%%%%%%%%%%%%%%%%%%%%%%%%%%%%%%%%%%%%%%%%%%%%%%%%%%%%%%%%%%%%%%

\setcounter{tocdepth}{1}
\setcounter{page}{1}
{\parskip=0mm \tableofcontents}
% \thispagestyle{empty}

\pagestyle{mine}

%%%%%%%%%%%%%%%%%%%%%%%%%%%%%%%%%%%%%%%%%%%%%%%%%%%%%%%%%%%%%%%%%%%%%%%%%%%%%%%%%%%%%%%%%%%%%
% Chapters
%%%%%%%%%%%%%%%%%%%%%%%%%%%%%%%%%%%%%%%%%%%%%%%%%%%%%%%%%%%%%%%%%%%%%%%%%%%%%%%%%%%%%%%%%%%%%
\cleardoublepage{}

%%%%%%%%%%%%%%%%%%%%%%%%%%%%%%%%%%%%%%%%%%%%%%%%%%%%%%%%%%%%%%%%%%%%%%%%%%%%%%%%%%%%%%%
\chapter{Introduction}
\label{chapter:Introduction}
%%%%%%%%%%%%%%%%%%%%%%%%%%%%%%%%%%%%%%%%%%%%%%%%%%%%%%%%%%%%%%%%%%%%%%%%%%%%%%%%%%%%%%%

Generally speaking, the introduction chapter should introduce the 
topic of the thesis and motivate the importance of it. Moreover, the
introduction should give an outline of the thesis and point out the 
contributions of this work.

You can logically group the chapters of the thesis
by using so-called parts. An example of how to insert a part-page 
containing a teaser-image is included in this template.
Typically, you will have an introductory chapter that gives a
broad overview. Then, the first part of the thesis might start after 
the introduction.


%%%%%%%%%%%%%%%%%%%%%%%%%%%%%%%%%%%%%%%%%%%%%%%%%%%%%%%%%%%%%%%%%%%%%%%%%%%%%%%%%%%%%%%
\chapter{Soft Dynamic Time Warping Algorithm}
\label{chapter:sdtw}
%%%%%%%%%%%%%%%%%%%%%%%%%%%%%%%%%%%%%%%%%%%%%%%%%%%%%%%%%%%%%%%%%%%%%%%%%%%%%%%%%%%%%%%


%%%%%%%%%%%%%%%%%%%%%%%%%%%%%%%%%%%%%%%%%%%%%%%%%%%%%%%%%%%%%%%%%%%%%%%%%%%%%%%%%%%%%%%
\section{Forward Pass}
\label{section:forward_pass}
%%%%%%%%%%%%%%%%%%%%%%%%%%%%%%%%%%%%%%%%%%%%%%%%%%%%%%%%%%%%%%%%%%%%%%%%%%%%%%%%%%%%%%%



%%%%%%%%%%%%%%%%%%%%%%%%%%%%%%%%%%%%%%%%%%%%%%%%%%%%%%%%%%%%%%%%%%%%%%%%%%%%%%%%%%%%%%%
\section{Backward Pass}
\label{section:backward_pass}
%%%%%%%%%%%%%%%%%%%%%%%%%%%%%%%%%%%%%%%%%%%%%%%%%%%%%%%%%%%%%%%%%%%%%%%%%%%%%%%%%%%%%%%

%%%%%%%%%%%%%%%%%%%%%%%%%%%%%%%%%%%%%%%%%%%%%%%%%%%%%%%%%%%%%%%%%%%%%%%%%%%%%%%%%%%%%%%
\section{Soft Alignment}
\label{section:soft_alignment}
%%%%%%%%%%%%%%%%%%%%%%%%%%%%%%%%%%%%%%%%%%%%%%%%%%%%%%%%%%%%%%%%%%%%%%%%%%%%%%%%%%%%%%%


%%%%%%%%%%%%%%%%%%%%%%%%%%%%%%%%%%%%%%%%%%%%%%%%%%%%%%%%%%%%%%%%%%%%%%%%%%%%%%%%%%%%%%%
\chapter{Experimental Setup}
\label{chapter:experimental_setup}
%%%%%%%%%%%%%%%%%%%%%%%%%%%%%%%%%%%%%%%%%%%%%%%%%%%%%%%%%%%%%%%%%%%%%%%%%%%%%%%%%%%%%%%
In this section, we present the objectives of our experiment, the dataset used for training alongside with the network architecture and the training process.

\section{Dataset}
\label{section:datset}
For the experiment, we use the Beatles dataset retrieved from Isophonics \cite{isopho}, consisting of four audio recordings with respective annotations. Due to the simplicity of the current chord recognition network, we need to simplify the chord labels, so that the annotations only have major and minor chords.
We split the dataset into training, validation, and test sets. For test set, a short segment of Let It Be is used, while the rest of the dataset is split into 3:1 ratio for training and validation.

We choose a sequence length of 150 samples for training and validating the model, creating 43 and 12 segments for training and validation, respectively.
In case of soft alignment, we remove the adjacent repetitions in the sequence, effectively reducing its length by around 85\% on average (see Figure \ref{figure:soft_alignment}).
However, this method introduce a problem where batching target sequences of different lengths is not possible. To address this issue, after reduction, we pad the sequences repeating each frame uniformly until they reach a desired length, or "soft length".
After some experiments, we found that a soft length of 16 covers all of possible reduced sequences while keeping the reduction ratio high.


\section{Model Architecture}
\label{section:model_architecture}
Given the aim of this experiment is to evaluate the performance of the proposed SDTW loss function, the network architecture plays a minor role and are kept simple.
Therefore, we used a simple chord recognition network (dChord) that based on the template-based chord recognition algorithm.
This network consists of a single layer that acts as the chord template to predict a 24-dimensional chord label activation vector, corresponding to 12 chromas with their respective major or minor variant. Combined with log-compression and feature normalization layers,
the network has a total of 25 trainable parameters.
Table \ref{table:arch} illustrate the components of the architecture with their respective input and output dimensions.

During training, we use Adam optimizer with a learning rate of 0.01.

\begin{table}[t]
    \begin{center}
    \begin{tabular}{|c|c|c|c|}
        \hline
        Layer & Input Dimension & Output Dimension & Parameters \\
        \hline
        Log-compression & (12, T) & (12, T) & 0 \\
        Normalization & (12, T) & (12, T) & 0 \\
        dChord & (12, T) & (24, T) & 25 \\
        softmax & (24, T) & (24, T) & 0 \\
        \hline
    \end{tabular}
        \caption{Architecture of the chord recognition network. T is the number of time frames.}
        \label{table:arch}
    \end{center}
\end{table}



\section{Results and Discussion}
\label{section:resultsdiscussion}


%%%%%%%%%%%%%%%%%%%%%%%%%%%%%%%%%%%%%%%%%%%%%%%%%%%%%%%%%%%%%%%%%%%%%%%%%%%%%%%%%%%%%%%
\chapter{Conclusions}
\label{chapter:conclusions}
%%%%%%%%%%%%%%%%%%%%%%%%%%%%%%%%%%%%%%%%%%%%%%%%%%%%%%%%%%%%%%%%%%%%%%%%%%%%%%%%%%%%%%%


% %%%%%%%%%%%%%%%%%%%%%%%%%%%%%%%%%%%%%%%%%%%%%%%%%%%%%%%%%%%%%%%%%%%%%%%%%%%%%%%%%%%%%%%%%%%%%
% % Appendix
% %%%%%%%%%%%%%%%%%%%%%%%%%%%%%%%%%%%%%%%%%%%%%%%%%%%%%%%%%%%%%%%%%%%%%%%%%%%%%%%%%%%%%%%%%%%%%
% \cleardoublepage{}
% \appendix
% %%%%%%%%%%%%%%%%%%%%%%%%%%%%%%%%%%%%%%%%%%%%%%%%%%%%%%%%%%%%%%%%%%%%%%%%%%%%%%%%%%%%%%%
% \chapter{Source Code}
% \label{chapter:source_code}
% %%%%%%%%%%%%%%%%%%%%%%%%%%%%%%%%%%%%%%%%%%%%%%%%%%%%%%%%%%%%%%%%%%%%%%%%%%%%%%%%%%%%%%%

% In this chapter, the headers of selected \MATLAB{} functions created during the writing of this thesis are reproduced. The headers contain information about the name of the described function and its input/output behavior.

% \section*{Feature Extraction}
% The \texttt{file\_to\_feature} function is used as a wrapper for several low-level functions that perform feature extraction or loading of precomputed features.

% Sample usage:\\
% \scriptsize
% \verb|[f_pitch, f_peaks] = file_to_feature('features', 'pathetique.wav');|

% \begin{verbatim}
% %%%%%%%%%%%%%%%%%%%%%%%%%%%%%%%%%%%%%%%%%%%%%%%%%%%%%%%%%%%%%%%%%%%%%%%%%%%
% % Name: file_to_feature
% % Version: 1.0
% % Date: 11.05.2010
% % Programmer: John Q. Public
% %
% % Description:
% %   Load or compute features for audio and MIDI files
% %
% % Input:
% % - dirname: Directory where the file or features are located
% % - filename: Name of the file for which to load/compute features
% % - parameter
% %            .win_len: Window length used for STMSP feature generation
% %            .win_res: Window resolution
% %
% % Output:
% % - f_pitch: Pitch features (STMSP)
% % - f_peaks: Energy peaks for onset computation
% % - f_onsets: Precise onsets (only generated in case of MIDI input data)
% %%%%%%%%%%%%%%%%%%%%%%%%%%%%%%%%%%%%%%%%%%%%%%%%%%%%%%%%%%%%%%%%%%%%%%%%%%%
% \end{verbatim}
% \normalsize

%%%%%%%%%%%%%%%%%%%%%%%%%%%%%%%%%%%%%%%%%%%%%%%%%%%%%%%%%%%%%%%%%%%%%%%%%%%%%%%%%%%%%%%%%%%%
% Bibliography
%%%%%%%%%%%%%%%%%%%%%%%%%%%%%%%%%%%%%%%%%%%%%%%%%%%%%%%%%%%%%%%%%%%%%%%%%%%%%%%%%%%%%%%%%%%%
\cleardoublepage{}
{
\small
\addcontentsline{toc}{chapter}{Bibliography}
\bibliographystyle{siam}
\bibliography{bibliography}
}
\cleardoublepage{}


\end{document}
